\section{Ricevimento}

\subsection{Discussione iniziale}

Il ricevimento è iniziato con alcuni chiarimenti riguardanti lo scambio di email avvenuto tra il gruppo e i professori circa lo stabilire una data per questo ricevimento e per la presentazione del PoC\footnote{Proof of Concept} nella prima revisione (RTB\footnote{Requirements and Technology Baseline}).

In seguito si è discussa la direzione ottimale da intraprendere per lo sviluppo del PoC. Lo scopo del PoC è quello di "costringere" a fare una selezione della parte tecnologica del progetto, studiando e verificando che ogni componente si integri bene. Il PoC deve, quindi, dimostrare che abbiamo scelto e approfondito in modo esaustivo le tecnologie scelte per soddisfare quest'ultima verifica.

\subsection{Domande e risposte}

\subsection*{È possibile cambiare tecnologia in seguito alla presentazione del PoC?}

\paragraph{Nota} L'idea iniziale era quella di procedere con Eclipse Mosquitto fino alla presentazione del PoC e in seguito sostituirla con altre tecnolgie, ad esempio HiveMQ.

No, è caldamente sconsigliato cambiare tecnologia dopo la presentazione del PoC, potrebbe essere segno di una pessima analisi/progettazione. Ogni modifica alle tecnologie deve essere \textbf{giustificata}.

\subsection*{Documento di \emph{Analisi dei requisiti} ideale?}
La parte fondamentale sono i requisiti. Devono essere atomici, quantitativi e verificabili. Specificare i \textbf{Requisiti di vincolo} e i \textbf{Requisiti di funzionalità}.

\subsection*{In che modo conviene (con che protocollo) far comunicare il server per fare in modo che la pagina dell'interfaccia utente si aggiorni automaticamente ad ogni cambiamento?}

Questa domanda deriva da VIN\_20230111.

\paragraph{Nota} Il problema principale è aggiornare in real-time le varie interfacce, mentre l'utente è collegato, ogni qualvolta il coordinatore decide di aggiornare lo stato di un lampione. Protocolli utili: XMTP, SOAP, Long polling, ecc.

Per ovviare a ciò la scelta più comune è l'impiego di WebSocket. Altre alternative potrebbero essere: Webhook, Server Sent Events (SSE).

\subsection*{Ci sono modi alternativi di rappresentare il \href{https://github.com/SWEasabi/analisi-dei-requisiti/blob/69fd3970d6618d506a725d9a7cd83afa3417eed1/src-doc/contenuti/img/casi_uso_grafici-applicazione\%2Cgestione.png}{Sistema di gestione (pag. 11)}?}

\emph{Rimandato a nuovo ricevimento/email.}