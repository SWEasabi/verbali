\section{Svolgimento}

\subsection{Modifiche da fare al documento di analisi}

All'analisi dei requisiti mancano sicuramente le seguenti sezioni:
\begin{itemize}
    \item Requisiti;
    \item User stories;
\end{itemize}
Che dovremo vedere di aggiungere.

Potrebbe essere interessante portare una descrizione testuale dei requisiti così come riportata dall'azienda.

Sarebbe inoltre da legare i verbali di riunioni con l'azienda al documento di analisi.

\subsection{Documentazioni per l'azienda}

All'azienda sarebbe da mostrare l'analisi dei requisiti con tutte le modifiche descritte in questo verbale, completate.

Con l'azienda sarebbe da discuterne l'aderenza al capitolato.

\subsection{Controllo completezza dei requisiti}

Nell'analisi dei requisiti manca tutta la lista dei requisiti scritti in maniera strutturata.

\subsection{Controllo delle funzionalità e degli UC}

Da sistemare: 

In uc04 attore principale è solo sensore guasti. Utente gestore è solo un sensore guasti con più funzionalità.

Bisogna trovarsi e discuterne meglio a voce la stesura generale.

L'uml ad ora fatto manca di dettaglio e sono presenti alcuni errori.

\subsection{Altre questioni emerse}

Ci siamo accorti che manca l'autocompilazione nel repository dell'analisi dei requisiti.

Manca l'immagine del grafico che descrive i casi d'uso nel documento dell'analisi dei requisiti.

Il portavoce del gruppo degli analisti potrebbe essere Alessandro.
