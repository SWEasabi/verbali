\section{Consuntivazione}

\subsection{Milestones:}
\begin{itemize}
    \item Release versione 1.0.0 del piano di qualifica (Completa al: 100\%)
    \item Release versione 1.0.0 delle norme di progetto (Completa al: 100\%)
    \item Release versione 1.0.0 del glossario (Completa al: 60\%)
\end{itemize}

\subsection{Attività svolte}

\begin{table}[H]
    \begin{xltabular}{\textwidth}{X l l}
        
        \rowcolor{gray!30} \textbf{Attività} & \textbf{Stato} & \textbf{Ruolo}\\
        \endhead
        \hline
        revisioni e correzioni finali ai documenti Piano di Progetto e Piano di Qualifica & completato & Analista \\
        rilascio versioni 1.0.0 Piano di Qualifica e Norme di Progetto & completato & Responsabile \\
        composizione del Glossario accorpando le sezioni “glossario” dei diversi documenti & parziale & Analista \\
    \end{xltabular}
    \caption{Lista delle attività svolte durante lo sprint}
\end{table}

\subsection{Trend e riflessioni}\label{subsec:trend}

Durante questo sprint non sono stati registrati problemi che hanno portato ad un rallentamento delle attività. E' stata però riscontrata una difficoltà ad organizzare un incontro che prevedesse la partecipazione di tutti i membri del gruppo.

Prevediamo nel prossimo sprint di recuperare il tempo perduto.

\subsection{Difficoltà e problemi di sprint}

Non ci sono state grandi difficoltà esterne durante questo sprint.

Internamente, come difficoltà, si rimanda alla riflessione posta nella sezione \ref{subsec:trend}.

Queste verranno discusse in sede di preparazione del prossimo sprint.
