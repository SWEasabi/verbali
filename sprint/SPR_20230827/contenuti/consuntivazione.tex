\section{Consuntivazione}

\subsection{Attività svolte}

\begin{table}[H]
    \begin{xltabular}{\textwidth}{X l l}

        \rowcolor{gray!30} \textbf{Attività} & \textbf{Stato} & \textbf{Ruolo}\\
        \endhead
        \hline
        rilascio versione 2.0.1 piano di progetto & completo & Responsabile \\
        rilascio versione 2.0.0 piano di qualifica & completo & Responsabile \\
        rilascio versione 1.0.0 specifica architetturale & completo & Responsabile \\
        rilascio versione 1.0.0 manuale utente & completo & Responsabile \\
        rilascio versione 1.0.1 glossario & completo & Responsabile \\
        redazione documento sull'analisi della collaborazione di gruppo & completo & Il gruppo \\
        stesura lettera di candidatura & completo & Il gruppo \\
        revisione semaforica & completo & Il gruppo \\
    \end{xltabular}
    \caption{Lista delle attività svolte durante lo sprint}
\end{table}

\subsection{Trend e riflessioni}\label{subsec:trend}

Durante questo sprint non sono stati registrati problemi che hanno portato ad un rallentamento delle attività, il gruppo ha cercato di risolvere i problemi che si erano manifestati nello sprint precedente in ottica revisione semaforica e revisione PB.

\subsection{Difficoltà e problemi di sprint}

Non ci sono state grandi difficoltà esterne durante questo sprint.

Internamente, come difficoltà, si rimanda alla riflessione posta nella sezione \ref{subsec:trend}.

Queste verranno discusse in sede di preparazione del prossimo sprint.