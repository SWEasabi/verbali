\section{Consuntivazione}

\subsection{Attività svolte}

\begin{table}[H]
    \begin{xltabular}{\textwidth}{X l l}

        \rowcolor{gray!30} \textbf{Attività} & \textbf{Stato} & \textbf{Ruolo}\\
        \endhead
        \hline
        stesura specifica delle basi di dati (appendice specifica architetturale) & completo & Progettista \\
        stesura lettera di candidatura alla revisione PB & completo & Responsabile \\
        sviluppo del sistema di anagrafe & completo & Programmatore \\
        sviluppo del sistema di coordinazione & parziale & Programmatore \\
        sviluppo del sistema di logging & parziale & Programmatore \\
        stesura della documentazione per il sistema di anagrafe & parziale & Progettista \\
        stesura della documentazione per il sistema di logging & parziale & Progettista \\
        stesura della documentazione per il sistema di coordinazione & parziale & Progettista \\
        stesura del manuale utente & parziale & Progettista \\
    \end{xltabular}
    \caption{Lista delle attività svolte durante lo sprint}
\end{table}

\subsection{Trend e riflessioni}\label{subsec:trend}

Durante questo sprint non sono stati registrati problemi che hanno portato ad un rallentamento delle attività, il gruppo ha cercato di lavorare il più possibile al fine di potersi candidare alla revisione PB.

\subsection{Difficoltà e problemi di sprint}

Non ci sono state grandi difficoltà esterne durante questo sprint.

Internamente, come difficoltà, si rimanda alla riflessione posta nella sezione \ref{subsec:trend}.

Queste verranno discusse in sede di preparazione del prossimo sprint.