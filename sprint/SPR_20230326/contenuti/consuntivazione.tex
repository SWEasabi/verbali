\section{Consuntivazione}

\subsection{Milestones:}
\begin{itemize}
    \item Release versione 0.0.6 dell'analisi dei requisiti (Completa al: 100\%)
    \item Release versione 0.1.0 dell'analisi dei requisiti (Completa al: 60\%)
    \item Release versione 0.1.1 dell'analisi dei requisiti (Completa al: 100\%)
    \item Release versione 1.0.0 dell'analisi dei requisiti (Completa al: 100\%)
\end{itemize}

\subsection{Attività svolte}

\begin{table}[H]
    \begin{xltabular}{\textwidth}{X l l}

        \rowcolor{gray!30} \textbf{Attività} & \textbf{Stato} & \textbf{Ruolo}\\
        \endhead
        \hline
        Stesura Piano di Qualifica e Piano di Progetto & parziale & Analista \\
        rilasciata versione 0.0.6 dell'analisi dei requisiti & completato & Responsabile \\
        rilasciata versione 0.1.0 dell'analisi dei requisiti & completato & Responsabile \\
        rilasciata versione 0.1.1 dell'analisi dei requisiti & completato & Responsabile \\
        rilasciata versione 1.0.0 dell'analisi dei requisiti & completato & Responsabile \\
        migliorata interfaccia grafica del \textbf{PoC} & completato & Analista \\
        finalizzazione documentazione per revisione \textbf{RTB} & parziale & Analista \\
    \end{xltabular}
    \caption{Lista delle attività svolte durante lo sprint}
\end{table}


\begin{table}[ht]
    \begin{tabularx}{\linewidth}{X|rrrrrrr}
    \rowcolor{gray!30}& Re & Amm & An & Pro & Prog & Ver & tot \\
    \hline
    Bonavigo Michele                        & 0 & 0,45 & 0 & 0 & 0 & 1,7  & 2,15 \\
    \rowcolor{gray!10}Casarotto Mattia      & 0 & 0,45 & 0 & 0 & 0 & 0,5 & 0,95 \\
    Massarenti Alessandro                   & 2 & 0 & 0 & 0 & 0 & 2,8  & 4,8 \\
    \rowcolor{gray!10}Peron Samuel          & 0 & 6,5 & 0 & 0 & 0 & 0 & 6,5 \\
    Pierobon Luca                           & 0 & 0 & 0 & 0 & 0 & 1 & 1 \\
    \rowcolor{gray!10}Romano Davide         & 0 & 0,5 & 0 & 0 & 0 & 0 & 0,5 \\
    Zarantonello Giorgio                    & 0 & 0 & 0 & 0 & 0 & 0 & 0 \\
    \hline                                  & 2 & 7,9 & 0,75 & 0 & 0 & 6 & \\
    \end{tabularx}
    \caption{\label{ruoli-persone}Spartizione dei ruoli e ore svolte durante lo sprint}
\end{table}

\begin{center}
\includegraphics[width=12cm]{img/ore-svolte.png}
\end{center}

\begin{table}[ht]
    \begin{tabularx}{\linewidth}{X|l|l}
    \rowcolor{gray!30}& Ore & Costo \\
    \hline

    Responsabile & 2 & € 60,00 \\
    \rowcolor{gray!10}Amministratore & 7,9 & € 158,00 \\
    Analista & 0 & € 0,00 \\
    \rowcolor{gray!10}Progettista & 0 & € 0,00 \\
    Programmatore & 0 & € 0,00 \\
    \rowcolor{gray!10}Verificatore & 6 &€ 90,00 \\
    totale & 15,9 & € 308,00 \\
    \end{tabularx}
    \caption{\label{costi-ruolo}Spartizione dei ruoli e ore svolte durante lo sprint}
\end{table}

Avendo quindi consumato €308,00\footnote{Si veda tabella \ref{costi-ruolo}} del budget durante questo sprint, rimangono ancora a disposizione € 9598,50 per gli sprint seguenti.

\subsection{Trend e riflessioni}\label{subsec:trend}

\begin{figure}[H]
    \includegraphics[width=\linewidth]{img/andamento.png}
    \caption{Andamento ore utilizzate nei vari sprint}\label{img:andamento}
\end{figure}

Durante questo sprint è stato registrato un calo della produttività generale, ma nonostante ciò si è comunque riusciti a progredire in vista della revisione RTB, con la presentazione al professor Cardin (che ha rilasciato il semaforo verde) di scelte tecnologiche, PoC e Analisi dei Requisi.

Prevediamo nel prossimo sprint di recuperare il tempo perduto.

\subsection{Difficoltà e problemi di sprint}

Non ci sono state grandi difficoltà esterne durante questo sprint. L'unica difficoltà riscontrata è stato il cambio dell supporto hardware, rispetto a quanto definito in sede di presentazione del capitolato, per i lampioni a cui dovremo fare fede.

Internamente, come difficoltà, si rimanda alla riflessione posta nella sezione \ref{subsec:trend}.

Queste verranno discusse in sede di preparazione del prossimo sprint.