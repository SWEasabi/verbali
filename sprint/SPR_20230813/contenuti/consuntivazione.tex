\section{Consuntivazione}

\subsection{Attività svolte}

\begin{table}[H]
    \begin{xltabular}{\textwidth}{X l l}

        \rowcolor{gray!30} \textbf{Attività} & \textbf{Stato} & \textbf{Ruolo}\\
        \endhead
        \hline
        sviluppo del sistema di coordinazione & completo & Programmatore \\
        sviluppo del sistema di logging & completo & Programmatore \\
        stesura versione 0.1.0 specifica architetturale & completo & Progettista\\
        correzione microservizi e documentazione per i sistemi di anagrafe, logging ed autenticazione dopo ragionamenti con il proponente in ottica MVP & completo & Progettista, Programmatore \\
        stesura del manuale utente & parziale & Progettista \\
        rilascio versione 0.1.0 specifica architetturale & completo & Responsabile \\
        visione di MVP con il proponente per ottenimento approvazione verso la visione semaforica & & Il gruppo \\
        formalizzazione della candidatura a revisione PB & & Il gruppo \\
    \end{xltabular}
    \caption{Lista delle attività svolte durante lo sprint}
\end{table}

\subsection{Trend e riflessioni}\label{subsec:trend}

Durante questo sprint sono stati registrati alcuni problemi che hanno portato ad un rallentamento delle attività. La problematica principale è stata la scarsità, o mancanza in alcuni casi, di collaborazione da parte di alcuni membri del gruppo, dettata anche da impegni personali a carico di alcuni membri. A questo poi si è aggiunto un problema di collaborazione tra i microservizi sviluppati, i quali hanno quindi richiesto degli aggiustamenti. Nonostante questi due problemi è stata comunque formalizzata la candidatura alla revisione PB prevista per il 21/08, ma il gruppo non è ovviamente soddisfatto della situazione venutasi a creare e si cercherà di fare tutto il possibile per far rientrare questo problema.

\subsection{Difficoltà e problemi di sprint}

Non ci sono state grandi difficoltà esterne durante questo sprint.

Internamente, come difficoltà, si rimanda alla riflessione posta nella sezione \ref{subsec:trend}.

Queste verranno discusse in sede di preparazione del prossimo sprint.